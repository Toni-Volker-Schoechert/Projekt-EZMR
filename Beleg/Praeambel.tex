%----- Schrift und Sprache --------------
\usepackage[ngerman]{babel}				%%% Sprachpaket für Deutsche Sprache

\usepackage[utf8]{inputenc}				%%% Verwendung von Umlauten und Kodierung dieser, sodass Latex es kompilieren kann

\usepackage[T1]{fontenc} 				%%% Silbentrennung bei Sonderzeichen

\usepackage{textcomp}					%%% Zusätzlihe Symbolzeichen

\usepackage{courier}					%%% Schriftart-Paket

\usepackage{lmodern}					%%% (=Lain Modern) Ausgabe von flüssigem, wenig verpixeltem Text
\usepackage{mathptmx} %%%times new roman ähnlich

\usepackage{romanbar}					%%% Verwendung römischer Zahlen

\usepackage{siunitx}					%%% Setzt Einheiten und Zahlen korrekt - nützlich besonders bei naturwissenschaftlichen Arbeiten

\usepackage{titlesec}

% Formatierung der Überschriften ohne Punkt nach den Nummern
\titleformat{\section}[block]{\normalfont\Large\bfseries}{\thesection}{1em}{}
\titleformat{\subsection}[block]{\normalfont\large\bfseries}{\thesubsection}{1em}{}
\titleformat{\subsubsection}[block]{\normalfont\normalsize\bfseries}{\thesubsubsection}{1em}{}


%------ Seite einrichten ----------------

\usepackage[includehead,includefoot, left=3.0cm, right=2.5cm, top=2.5cm, bottom=2.5cm]{geometry}								%%% Gestaltung der Seitenrändern, Kopf- und Fußzeile

\usepackage[onehalfspacing]{setspace} 	%%% Zeilenbstand anpassen, Hier:1,5-fach
\usepackage{listings}
\usepackage{xcolor}
\usepackage{courier}
\lstset{
  basicstyle=\ttfamily\footnotesize,
  breaklines=true,        % Automatischer Zeilenumbruch
  breakatwhitespace=false % Falls gewünscht, nicht nur an Leerzeichen umbrechen
}
\usepackage[utf8]{inputenc}
\usepackage[T1]{fontenc}
\lstset{
  literate=
    {ä}{{\"a}}1
    {ö}{{\"o}}1
    {ü}{{\"u}}1
    {Ä}{{\"A}}1
    {Ö}{{\"O}}1
    {Ü}{{\"U}}1
    {ß}{{\ss}}1
}


\usepackage{scrextend}					%%% Erweitert die Möglichkeiten von bestimmten Dokumentklassen

\setlength{\parindent}{0pt}             %%% Nach Zeilenumbruch wird die 1. Zeile im neuen Abschnitt nicht eingerückt

%----- Mathemaische Formeln ------------
\usepackage{amsmath} 					%%% Erweitert den möglichen Formelsatz
\usepackage{amssymb}					%%% Fügt weiter mathematische Symbole und Pfeile ein

%----- Tabellen ------------------------

\usepackage{booktabs}					%%% Tabellen formatieren - horizotale Linien

\usepackage{multirow}					%%% Tabellen foratieren - mehrzeiliger Text in einer Zelle

\usepackage{colortbl}					%%% Um Zellen in Tabellen einzufärben
\usepackage{xcolor} 					%%% Bindet ein Farb-Paket ein 

%%% Vergrößerung des Platzes in Tabellen, besonders nach \hline notwenig
\newcommand\T{\rule{0pt}{2.6ex}}       	% oben
\newcommand\B{\rule[-1.2ex]{0pt}{0pt}} 	% unten  

\usepackage{subfigure}					%%% Zur beschriftung mehrerer Bilder mit dem gleichen Titel

\usepackage{float}						%%% Biler und Tabellen werden horizontal umflossen

\usepackage[export]{adjustbox}			%%% Ausrichtung von Figuren links, rechts, zentriert, innen und außen

%----- Zitieren und Referenzen -----------------

\usepackage[round]{natbib}				%%% Zitationstil festlegen (Hier: Autor, Jahr)

\usepackage[german=quotes]{csquotes} 	%%% korrektes Anzeigen von Zitaten

\usepackage{makeidx}
\makeindex
%----- Hyperlinks -----------------------------
\usepackage{hyperref}					%%% Einbinden von Links und Verweisen im Dokument
\hypersetup{colorlinks=true} 			%%% Ermöglicht das einfärben von Links
\hypersetup{urlcolor=blue}				%%% URL werden blau angezeigt
\hypersetup{citecolor=black}			%%% Zitate werden schwarz angezeigt
\hypersetup{linkcolor=black}			%%% Verweise werden schwarz angezeigt

%----- Importieren -----

\usepackage{graphicx}					%%% Einbinden von Graiken und Bildern

\usepackage{wrapfig}					%%% um Grafiken und Bilder textumflossen einzubinden - muss als letztes Paket eingebunden werden, da es zum Teil auf andere Pakete zugreift
\usepackage{pdfpages}					%%% PDF können importiert werden



%%%%%%%%%%%%%%%%%%%%%%%%%%%%%%%%%%%%%%%%%%%%%%%%%%%%%%%%%%%%%%%%%%%%%%%%%%%%%%%%%%%%%%
% Alles, was jetzt folgt definiert Kopf- und Fußzeilen neu und bindet so das HTWK-Design ein.
%
%\clearscrheadfoot					% Löscht alle Kopf- und Fußzeilen
%\setfootwidth{\textwidth}			% Definiert die Breite von Kopf und Fußzeile
%\ohead{\hspace{-6cm} \raisebox{6cm}{\includegraphics[width=10cm]{GELB} }}
%									%  Fügt das gelbe Rechteck im äußeren Bereich der Fußzeile ein
%\ihead{\raisebox{10cm}{\headmark}}	% Fügt im inneren Bereich klein die Kapitelüberschriften ein
%\ofoot{\raisebox{\baselineskip}{\large {\textbf{\pagemark}}}}%[24mm][12mm]
%									% Setzt unten, außen die Seitenzahl

%Programmablaufplan
\usepackage{tikz}
\usetikzlibrary{shapes.geometric, arrows}
\usetikzlibrary{shapes}
\usetikzlibrary{positioning}
\usetikzlibrary{circuits.ee.IEC}

\usepackage{adjustbox} %besseres drehen der bilder

\usepackage{pdfpages}%%pdf einfügen

%%%Matlab code
\definecolor{mymauve}{rgb}{0.58,0,0.82}

\lstset{%
  language=Matlab,
  basicstyle=\ttfamily\small,
  keywordstyle=\color{blue},
  commentstyle=\color{gray},
  stringstyle=\color{mymauve},
  breaklines=true,
  showstringspaces=false,
  frame=single
}

% Pakete für Tabellen und Listings
\usepackage{tabularx}
\usepackage{wrapfig}
\usepackage{multirow}
\usepackage{listings}


% Definitionen für Tabellen
\newcolumntype{x}[1]{!{\centering\arraybackslash\vrule width #1}}
\newcolumntype{C}[1]{>{\centering\arraybackslash}p{#1}}
\setlength{\aboverulesep}{0pt}
\setlength{\belowrulesep}{0pt}
\renewcommand*{\arraystretch}{1.2}

% Einstellungen für Listings
\lstset{
	frame=single,
	language=C,
	keywordstyle=\color{blue},
	commentstyle=\color{gray},
	numbers=left,
}


