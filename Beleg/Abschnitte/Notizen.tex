Vorschlag zur Gliederung und Inhalt für deinen Beleg
1 Einleitung

    Motivation: Warum machst du das Praktikum und die Programmieraufgaben? Was ist der Nutzen?

    Aufgabenstellung: Kurze Übersicht, was gemacht werden soll (z.B. Netzwerksockets programmieren).

    Aufbau der Arbeit: Kurz erklären, wie die Arbeit strukturiert ist.

2 Programmieraufgaben
2.1 Sockets/Netcode

    2.1.1 Teilaufgabe a)

        Beschreibung der Aufgabe (z.B. TCP-Server erstellen)

        Vorgehensweise / Umsetzung (Beschreibung wie du den Server programmiert hast)

        Erläuterung des Codes (wichtige Funktionen, Bibliotheken, Vorgehen)

        Tests / Ergebnisse (z.B. mit netcat getestet)

        Probleme und Lösungen

        Fazit / Erkenntnisse

(Falls weitere Teilaufgaben kommen, kommen sie hier als 2.1.2, 2.1.3, …)
3 Praktikum

    3.1 Beschreibung des Praktikums

    3.2 Aufgaben im Praktikum

    3.3 Erlernte Fähigkeiten / Erkenntnisse

    3.4 Herausforderungen / Lösungen

    3.5 Fazit zum Praktikum

5 Quellen

    Literatur, Man-Pages, Webseiten, Tutorials etc.

6 Anhang (optional)

    Vollständiger Code

    Screenshots

    Testprotokolle

Tipps

    Jede Aufgabe in einen eigenen Unterabschnitt (z.B. 2.1.1, 2.1.2)

    Erkläre nicht nur was, sondern auch warum du so vorgegangen bist

    Nutze Bilder oder Diagramme, falls sinnvoll

    Halte das Inhaltsverzeichnis aktuell, das erleichtert die Navigation

    Das Abbildungsverzeichnis brauchst du nur, wenn du viele Bilder/Grafiken hast


    1. Code selbst Schritt für Schritt durchgehen

    Lies dir den Code abschnittsweise durch (z.B. Zeile 1–20, dann 21–40 usw.).

    Überlege:

        Was macht dieser Abschnitt?

        Welche Funktionen werden verwendet?

        Warum ist das nötig?

        Wie hängen die Teile zusammen?

2. Man-Pages parallel nutzen

    Suche für jede unbekannte Funktion die passende Man-Page (man 2 socket, man 2 bind usw.).

    Lies die Beschreibung und probiere zu verstehen, welche Argumente die Funktion erwartet und was sie macht.

    Notiere wichtige Punkte (kurz und prägnant).

3. Schreibe eigene kurze Erklärungen zu jedem Schritt

Zum Beispiel:

    „Der Server erstellt zuerst mit socket() einen TCP/IP Socket. Das ist die Kommunikationsschnittstelle für das Netzwerk.“

    „Mit bind() wird der Socket an die Adresse und den Port gebunden, damit Verbindungen akzeptiert werden können.“

    „listen() versetzt den Socket in den Zustand, auf eingehende Verbindungen zu warten.“

    „accept() blockiert, bis ein Client verbindet, und gibt einen neuen Socket für den Dialog zurück.“

    „Daten werden mit recv() gelesen und mit send() eine Antwort gesendet.“

4. Teste den Code selbst

    Führe das Programm aus, spiele mit netcat oder telnet rum.

    Probiere verschiedene Eingaben aus.

    Beobachte, was passiert und notiere es.

5. Visualisiere den Ablauf (optional)

    Zeichne eine kleine Skizze:

        Server wartet → Client verbindet → Daten senden/empfangen → Verbindung schließen

    Das hilft beim Verstehen und Erklären.

6. Formuliere für den Beleg

    Nutze deine eigenen Notizen, um klar und verständlich zu beschreiben, wie das Programm funktioniert.

    Vermeide es, einfach den Code zu kopieren, sondern schreibe in eigenen Worten.

    Ergänze wichtige technische Details, z.B. warum htons() nötig ist, oder was INADDR_ANY bedeutet.

7. Fragen stellen und klären

    Wenn etwas unklar ist, frag hier oder suche Beispiele im Netz.

    Verstehe auch, was passiert, wenn etwas schiefgeht (z.B. was bind() bei Fehlern macht).


    