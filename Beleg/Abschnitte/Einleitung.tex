\chapter{Einleitung}

Im Rahmen des Moduls Echtzeitsysteme und Mobile Robotik wurden zwei größere praktische Aufgaben bearbeitet. Ziel war es, grundlegende Konzepte der Prozesskommunikation sowie den Aufbau und die Funktionsweise von Robotersystemen mit ROS 2 zu verstehen und eigenständig umzusetzen. Beide Themen wurden durch eigenständige Programmierung, Tests und Dokumentation praktisch vertieft.

Der erste Teil beschäftigte sich mit der Kommunikation zwischen Prozessen und Anwendungen. Dazu wurden Programme entwickelt, die mithilfe von \textit{Sockets} und  \textit{Interprozesskommunikation (IPC)} Daten austauschen können. Die Aufgaben umfassten die Erstellung von Server- und Client-Programmen, den Aufbau von Verbindungen über IP und Port sowie die Verwaltung gemeinsamer Datenstrukturen. Ziel war es, ein Verständnis dafür zu entwickeln, wie Prozesse in Echtzeitsystemen Informationen austauschen.

Im zweiten Teil stand die praktische Arbeit mit dem Robot Operating System 2 (ROS 2) im Mittelpunkt. Zunächst wurden die Grundlagen anhand der turtlesim-Simulation erlernt, um die Kommunikation zwischen Publishern und Subscribern zu verstehen. Anschließend wurde ein eigenes ROS-Paket entwickelt, das Bewegungsbefehle an die Turtle sendet und einfache Fahrbewegungen automatisch ausführt. Dieses Wissen wurde danach auf einen realen mobilen Roboter übertragen, der mit einem ESP32-Mikrocontroller ausgestattet ist und über micro-ROS mit dem ROS-Netzwerk kommuniziert. Der Roboter wurde so programmiert, dass er auf Bewegungsbefehle reagiert und Sensordaten erfassen kann. Abschließend wurde mit dem TurtleBot 4 gearbeitet, um Funktionen wie Kartierung, Navigation und autonome Bewegung kennenzulernen.

Der Beleg dokumentiert die Umsetzung beider Aufgaben, beschreibt die wichtigsten technischen Schritte und fasst die gewonnenen Erkenntnisse zusammen. Außerdem werden Herausforderungen und Lösungsansätze beschrieben, die während der Bearbeitung aufgetreten sind. Der vollständige Quellcode der Programme ist in einem öffentlichen GitHub-Repository \ref{link:git} abgelegt und ergänzt die hier dargestellten Ergebnisse.
